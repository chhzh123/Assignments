\documentclass[logo,reportComp]{thesis}
\usepackage[python,pseudo]{mypackage}

\title{机器学习与数据挖掘大作业}
\subtitle{多标签用户人格分类(集成学习)}
\school{数据科学与计算机学院}
\author{陈鸿峥}
\classname{17大数据与人工智能}
\stunum{17341015}
\headercontext{机器学习与数据挖掘大作业}

\let\emph\relax % there's no \RedeclareTextFontCommand
\DeclareTextFontCommand{\emph}{\kaiti\em}

\begin{document}

\maketitle
\tableofcontents

\newpage
MBTI理论认为一个人的个性可以从四个角度进行分析,用字母代表如下:
\begin{itemize}
    \item 驱动力的来源:外向E---内向I
    \item 接受信息的方式:感觉S---直觉N
    \item 决策的方式:思维T---情感F
    \item 对待不确定性的态度:判断J---知觉P
\end{itemize}
按照不同的组合,可以产生16种人格类型。

本次大作业要求利用机器学习方法,通过用户的发言记录对用户的人格类型进行分类\footnote{数据集链接:\url{https://www.kaggle.com/datasnaek/mbti-type}}。

本实验报告为大作业的第一部分---集成学习方法。

\section{预处理}
emoji :sad:
移除网页超链接

\section{集成模型}
\subsection{数据生成}
\subsection{模型搭建}
\subsection{模型训练}
\subsection{模型预测}
\subsection{其他实现细节}

\section{实验结果}
\subsection{超参数选择}
\subsection{综合比较}
\section{总结与思考}
充分利用上学期自然语言处理课程的知识。

\begin{thebibliography}{99}

\end{thebibliography}

\end{document}

% 作业内容:
% 1. 使用集成学习方法完成人格分类(6月10日)
% * 对数据进行预处理
% * 使用集成学习模型(AdaBoost或Random Forest)进行人格分类
% * 提交报告及代码
% 2. 使用SVM进行人格分类(6月30日)
% * 数据预处理
% * 使用SVM进行人格分类
% * 提交报告及代码
% 3. 使用深度学习模型进行人格分类(7月31日)
% * 数据预处理
% * 使用深度学习模型(不限)进行人格分类
% * 提交代码和报告
% ** 深度学习方法需要有方法上的创新,如果只简单使用开源代码或框架,最多只能拿8分
% ** 加分项:使用英文撰写报告,设计合理实验(对比不同模型表现/不同超参数对性能的影响),自己撰写模型代码,尽可能少调用工具

% 评价指标:
% * 单独对每种类别进行评价/整体评价
% * F1 \& Accuracy

% 评分:
% * 集成学习方法:8分
% * SVM方法:8分
% * 深度学习方法:14分