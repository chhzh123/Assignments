\documentclass[reportComp]{thesis}
\usepackage[cpp,pseudo]{mypackage}

\title{模式识别作业一}
\subtitle{}
\school{数据科学与计算机学院}
\author{陈鸿峥}
\classname{17大数据与人工智能}
\stunum{17341015}
\headercontext{模式识别作业}
\lstset{language=python}

\begin{document}

\maketitle

\begin{question}[\textsection 2 Q2]
假设两个等概率的一维密度具有如下形式:对任给$i=1,2$及$0<b_i$,$p(x\mid\omega_i)\propto\ee^{-|x-a_i|/b_i}$。
\begin{itemize}
	\item [(a)] 写出每个密度的解析表达式,即对任意的$a_i$和正的$b_i$,将每个函数归一化
	\item [(b)] 计算似然比,作为$4$个变量的函数
	\item [(c)] 绘出在$a_1=0,b_1=1,a_2=1,b_2=2$时的似然比$p(x\mid\omega_1)/p(x\mid\omega_2)$的曲线图
\end{itemize}
\end{question}
\begin{answer}
\begin{itemize}
	\item [(a)] 设比例系数为$k_i$,由概率的基本性质有
\[\begin{aligned}
\qquad&\intab{-\infty}{\infty}{k_i\ee^{\frac{-|x-a_i|}{b_i}}}\\
=&\intab{-\infty}{a_i}{k_i\ee^{\frac{x-a_i}{b_i}}}+\intab{a_i}{\infty}{k_i\ee^{-\frac{x-a_i}{b_i}}}\\
=&2k_ib_i\\\
=&1
\end{aligned}\]
进而求得$k_i=1/(2b_i)$,故解析表达式为
\[p(x\mid\omega_i)=\frac{1}{2b_i}\ee^{-|x-a_i|/b_i}\]
	\item [(b)]
\[\frac{p(x\mid\omega_1)}{p(x\mid\omega_2)}=\frac{b_2}{b_1}\ee^{-\frac{|x-a_1|}{b_1}+\frac{|x-a_2|}{b_2}}\]
	\item [(c)]
将$a_1=0,b_1=1,a_2=1,b_2=2$代入(b)求得的式子化简得
\[\frac{p(x\mid\omega_1)}{p(x\mid\omega_2)}=2\ee^{-|x|+\frac{|x-1|}{2}}\]
图像如下
\begin{figure}[H]
\centering
\includegraphics[width=0.5\linewidth]{likelihood.pdf}
\end{figure}
\end{itemize}
\end{answer}

\begin{question}[\textsection 2 Q7]
考虑两个一维柯西分布的Neyman-Pearson准则:
\[p(x\mid\omega_i)=\frac{1}{\pi b}\cdot\frac{1}{1+\lrp{\frac{x-a_i}{b}}^2}\,,\qquad i=1,2\]
在0-1误差损失下,且为了简化,设$a_2>a_1$,宽度$b$相同,且先验概率相等。
\begin{itemize}
	\item [(a)] 假设当一样本实际属于$\omega_1$却被误认为$\omega_2$的模式分类时的最大可接受误差率为$E_1$,用所给变量确定判决边界。
	\item [(b)] 对于此边界,将$\omega_2$错分为$\omega_1$的误差率是多少?
	\item [(c)] 在0-1损失率下的总误差率是多少?
	\item [(d)] 将你的结论应用于特殊情况:$b=1$且$a_1=-1,a_2=1$且$E_1=0.1$
	\item [(e)] 将你的结论与贝叶斯误差率(即没有Neyman-Pearson条件)作比较。
\end{itemize}
\end{question}
\begin{answer}
\begin{itemize}
	\item [(a)] 因先验概率相等,故$p(x\mid\omega_1)=1/2$。
	设判别边界为$x^\star$,则
	\[\begin{aligned}
	E_1&=\intab{x^\star}{\infty}{p(x\mid\omega_1)P(\omega_1)}\\
	&=\frac{1}{2}\intab{x^\star}{\infty}{\frac{1}{\pi b}\frac{1}{1+\lrp{\frac{x-a_1}{b}}^2}}\\
	&=\frac{1}{2\pi b}\intabu{u^\star}{\infty}{\frac{b}{1+u^2}}{u}\qquad u=\frac{x-a_1}{b}\\
	&=\frac{1}{2\pi}\arctan u\Big|_{u^\star}^{\infty}
	\end{aligned}\]
	移项得到
	\[2\pi E_1=\frac{\pi}{2}-\arctan\frac{x^\star-a_1}{b}\]
	两侧同时取$\tan$,整理得
	\[x^\star=a_1+\frac{b}{\tan(2\pi E_1)}\]

	\item [(b)] 由题(a)的结果
	\[\begin{aligned}
	E_2&=\intab{-\infty}{x^\star}{p(x\mid\omega_2)P(\omega_2)}\\
	&=\frac{1}{\pi b}\intab{-\infty}{x^\star}{\frac{1}{1+\lrp{\frac{x-a_i}{b}}^2}P(\omega_2)}\\
	&=\frac{1}{2\pi b}\intabu{-\infty}{u^\star}{\frac{b}{1+u^2}}{u}\qquad u=\frac{x-a_2}{b}\\
	&=\frac{1}{2\pi}\arctan u\Big|_{-\infty}^{u^\star}\\
	&=\frac{1}{4}+\frac{1}{2\pi}\arctan\frac{x^\star-a_2}{b}
	\end{aligned}\]
	
	\item [(c)] 综合题(a)和题(b)有
	\[E=E_1+E_2=E_1+\frac{1}{4}+\frac{1}{2\pi}\arctan\frac{x^\star-a_2}{b}\]
	
	\item [(d)] 将值代入(c)式得到$x^\star=0.3764$,$E=0.2613$
	
	\item [(e)] 贝叶斯的决策边界为$x_B^\star=0$,进而贝叶斯误差率为
	\[\begin{aligned}
	E_B&=2\cdot\frac{1}{2\pi b}\intab{0}{\infty}{\frac{1}{1+\lrp{\frac{x-a_1}{b}}^2}}\\
	&=\arctan u\Big|_0^\infty\\
	&=0.25<E
	\end{aligned}\]
	即确实贝叶斯误差率会小于基于Neyman-Pearson准则的误差率
\end{itemize}
\end{answer}

\begin{question}[\textsection 2 Q9]
使用第7题给出的条件密度,设类别的先验概率相等。
\begin{itemize}
	\item [(a)] 证明最小误差概率为
	\[P(error)=\frac{1}{2}-\frac{1}{\pi}\tan^{-1}\left|\frac{a_2-a_1}{2b}\right|\]
	\item [(b)] 绘出它随$|a_2-a_1|/(2b)$变化的曲线图。
	\item [(c)] $P(error)$的最大值是多少?在什么条件下可以达到此值?试说明原因。
\end{itemize}
\end{question}
\begin{answer}
\begin{itemize}
	\item [(a)] 不妨设$a_2>a_1$,判决边界为$(a_1+a_2)/2$,则误差概率为
	\[\begin{aligned}
	P(error)&=\intab{-\infty}{(a_1+a_2)/2}{p(x\mid \omega_2)P(\omega_2)}+\intab{(a_1+a_2)/2}{\infty}{p(x\mid \omega_1)P(\omega_1)}\\
	&=\frac{1}{2\pi b}\lrp{\intab{-\infty}{(a_1+a_2)/2}{\frac{1}{1+\lrp{\frac{x-a_2}{b}}^2}}+\intab{(a_1+a_2)/2}{\infty}{\frac{1}{1+\lrp{\frac{x-a_1}{b}}^2}}}\\
	&=\frac{1}{2\pi b}\lrp{\arctan u\Big|_{-\infty}^{(a_1-a_2)/(2b)}+\arctan u\Big|_{(-a_1+a_2)/(2b)}^{\infty}}\\
	&=\frac{1}{2}-\frac{1}{\pi}\arctan\frac{a_2-a_1}{2b}
	\end{aligned}\]
	对于$a_1\leq a_2$的情形类似,故得证

	\item [(b)] 如下图所示
	\begin{figure}[H]
	\centering
	\includegraphics[width=0.5\linewidth]{error.pdf}
	\end{figure}

	\item [(c)] 当$|a_2-a_1|/(2b)=0$时$P(error)$达到最大值$1/2$,也即两个概率分布相同($a_1=a_2$),或者两个概率分布都为常量值($a_1\ne a_2,b=\infty$)
\end{itemize}
\end{answer}

\end{document}
% ftp://222.200.180.156/
% student
% 2019s