\documentclass[11pt,UTF8]{ctexart}
\usepackage{titling}
\usepackage{enumerate}
\usepackage{amsmath}
\usepackage{amssymb}
\usepackage{amsfonts}
\usepackage{listings}
\usepackage{forest}
\usepackage{bm}
\usepackage{float}
\usepackage{graphicx}
\usepackage{multicol}
\usepackage[unicode=true,%本行非常重要 支持中文目录hyperref CJKbookmarks对二级目录没用
	colorlinks,
	linkcolor=black,
	anchorcolor=black,
	citecolor=black,
	CJKbookmarks=false]{hyperref}
\usepackage{xcolor}
\usepackage{geometry}
\geometry{top=20mm,bottom=20mm,left=20mm,right=20mm}
\pagestyle{plain}%删除页眉
\CTEXsetup[format={\Large\bfseries}]{section}

\setlength{\droptitle}{-50pt}%减少标题与页眉距离

\title{Project 1 技术报告}
\author{17341015 计一 陈鸿峥}
\date{}

\begin{document}
\maketitle
\vspace{-50pt}%减少标题与正文距离

\lstset{language=C++,escapechar=`}

\section{需求分析}
仓库管理系统的功能如下:
\begin{enumerate}
	\itemsep -3pt
	\item 在仓库进货时,如果仓库中没有此商品,则为仓库增添新的商品项目
	\item 在仓库进货时,如果仓库中已有此商品,则增加此商品的库存量
	\item 在仓库出货时,减少对应商品的库存量
	\item 在仓库出货时,如果这是货物是此商品的最后一批货(库存量为0),则删除仓库中此商品项目
	\item 查询功能:可以随时查看当前仓库的库存,包括商品名和剩余量
\end{enumerate}
\par 且以上功能均需通过日志文件保存,以防丢失。


\section{实现思路}
\begin{center}
\begin{forest}
for tree = {grow=south}
[仓库管理系统
	[进货
		[增加库存]
		[新增商品]]
	[出货
		[减少库存]
		[删除商品]]
	[查询
		[显示库存]
		[查找商品]]]
\end{forest}
\end{center}
\par 本仓库管理系统主要分为进货、出货、查询三大模块,分别对进货、出货、查询的操作进行管理。
\begin{enumerate}
	\item 进货模块\\
	细分为增加库存和新增商品子功能。进货时,若此商品在仓库中没有库存,则在仓库库存条目中新增此商品项目;若已有此类商品,则根据进货量增加对应的库存量。
	\item 出货模块\\
	细分为减少库存和删除商品子功能。出货时,若此商品在仓库中没有库存,则报错显示出货失败;若此类商品库存充足,则根据出货量减少对应商品的数目;若出货成功并且库存恰好为$0$时,删除仓库目录中此商品项目。
	\item 查询模块\\
	细分为显示所有库存和查找某一商品子功能。
\end{enumerate}
\par 最后用文件流实现数据的读取及保存。


\section{数据设计}
\begin{lstlisting}
class Warehouse
{
public:
	Warehouse(){};
	bool import_goods(string name, int count);
	bool export_goods(string name, int count);
	int find_goods(string name);
	void show_goods();
	void save();
private:
	map <string,int> data;
	bool increase_count(string name, int count);
	bool add_to_list(string name, int count);
	bool decrease_count(string name, int count);
	bool delete_from_list(string name);
};
\end{lstlisting}
\par 所有商品数据都被存储在一个 \verb'Warehouse' 类中,且以 \verb'map' 形式存储,其中 \verb'map' 是由商品名字( \verb'string' 类型)到商品库存量( \verb'int' 类型)的映射。至于类方法,只有进货、出货、查询、保存是公有的,其他的方法均设置为私有。


\section{函数设计}
\par 详情请参见代码,这里仅仅给出每个函数的用途。
\subsection{进货}
\begin{enumerate}
	\item 进货,对应进货模块,表示当前进货一批数量为count的name商品\\
\verb'bool Warehouse::import_goods(string name, int count)'
	\item 更新库存信息,对应增加库存子功能,对name商品新增count数量\\
\verb'bool Warehouse::increase_count(string name, int count)'
	\item 更新库存列表,对应新增商品子功能,新增name商品且初始数量为count\\
\verb'bool Warehouse::add_to_list(string name, int count)'
	\item 封装函数\\
\verb'void import_process(Warehouse& house)'
\end{enumerate}

\subsection{出货}
\begin{enumerate}
	\item 出货,对应出货模块,表示当前出货一批数量为count的name商品\\
\verb'bool Warehouse::export_goods(string name, int count)'
	\item 更新库存信息,对应减少库存子功能,对name商品减少count数量\\
\verb'bool Warehouse::decrease_count(string name, int count)'
	\item 更新库存列表,对应删除商品子功能,删除商品列表中name商品\\
\verb'bool Warehouse::delete_from_list(string name)'
	\item 封装函数\\
\verb'void export_process(Warehouse& house)'
\end{enumerate}

\subsection{查询}
\begin{enumerate}
	\item 显示当前库存列表,包括商品名及其库存量\\
\verb'void Warehouse::show_goods()' 
	\item 查看仓库中名为name的商品,若不存在返回$0$,否则返回该商品库存量\\
\verb'int Warehouse::find_goods(string name)'
	\item 封装函数\\
\verb'void query_process(Warehouse& house)'
\end{enumerate}

\subsection{输入与输出}
\begin{enumerate}
	\item 从文档中读入数据\\
\verb'void read_in_data(Warehouse& house)'
	\item 保存数据入文档\\
\verb'void Warehouse::save()'
\end{enumerate}

\subsection{其他}
\begin{enumerate}
	\item 交互界面\\
\verb'void warehouse_interface()'
	\item 显示菜单\\
\verb'void show_manual()'
	\item 商品名合法性测试\\
\verb'bool validname_test(string x)'
\end{enumerate}



\end{document}