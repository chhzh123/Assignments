\documentclass[logo,reportComp]{thesis}
\usepackage[cpp,pseudo]{mypackage}

\title{编译原理作业一}
\subtitle{}
\school{数据科学与计算机学院}
\author{陈鸿峥}
\classname{17大数据与人工智能}
\stunum{17341015}
\headercontext{编译原理作业}

% Apr 21 -> Apr 25

\begin{document}

\maketitle

\begin{question}
一、	下列正则表达式定义了什么语言(用尽可能简短的自然语言描述)?
\begin{enumerate}
	\item $b^*(ab^*ab^*)^*$
	\item $c^*a(a|c)^*b(a|b|c)^*|c^*b(b|c)^*a(a|b|c)^*$
\end{enumerate}
\end{question}
\begin{answer}
写出一些例子后整理可知
\begin{enumerate}
	% aa, aabbbb, abbba, abbabbb, ababbaba
	\item 所有包含偶数个$a$的由$a$和$b$组成的字符串
	% cacba, cbcaa
	\item 所有至少包含$1$个$a$和$1$个$b$的由$a$、$b$、$c$组成的字符串
\end{enumerate}
\end{answer}

\begin{question}
设字母表$\Sigma=\{a,b\}$,用正则表达式(只使用$a$,$b$,$\epsilon$,$|$,${}^*$,$+$,$?$)描述下列语言:
\begin{enumerate}
	\item 不包含子串$ab$的所有字符串.
	\item 不包含子串$abb$的所有字符串.
	\item 不包含子序列$abb$的所有字符串.
\end{enumerate}
注意:关于子串(substring)和子序列(subsequence)的区别可以参考课本第119页方框中的内容.
\end{question}
\begin{answer}
写出一些特殊样例后整理可得
\begin{enumerate}
	% eps, a.., b.., b..a..
	\item 即所有$b$都在$a$前面,\\$b^*a^*$
	% (a*b?)* -> x a b b
	\item 即至多有一个$b$紧跟着$a$,\\$b^*(ab?)^*$
	% a*, b*, a*ba*, b*a*, b*a*b
	% a*b?a*|b*a*b?
	\item 即至多有一个$b$在$a$后面,\\$a^*b?a^*|b^*a^*b?$或$b^*a^*b?a^*$
\end{enumerate}
\end{answer}

\end{document}