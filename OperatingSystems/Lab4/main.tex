\documentclass[logo,reportComp]{thesis}
\usepackage[cpp,pseudo]{mypackage}

\title{操作系统原理实验报告}
\subtitle{实验四:中断处理与系统调用}
\school{数据科学与计算机学院}
\author{陈鸿峥}
\classname{17大数据与人工智能}
\stunum{17341015}
\headercontext{操作系统原理实验报告}
% \authorremark{本实验报告用\LaTeX撰写,创建时间:\builddate\today}

\begin{document}

\maketitle

\section{实验目的}
\begin{itemize}
	\item 学习中断中断机制知识,掌握中断处理程序设计的要求
	\item 学习通过汇编程序实现时钟中断处理
\end{itemize}

\section{实验要求}
% 实验目的和实验要求由老师提供实验项目文档中获取
\begin{itemize}
	\item 操作系统工作期间,利用时钟中断,在屏幕最边缘处动态画框,第一次用字母A,第二次画用字母B,如此类推,还可加上变色闪耀等效果。适当控制显示速度,以方便观察效果。
	\item 编写键盘中断响应程序,原有的你设计的用户程序运行时,键盘事件会做出有事反应:当键盘有按键时,屏幕适当位置显示``OUCH!OUCH!''。
	\item 在内核中,对33号、34号、35号和36号中断编写中断服务程序,分别在屏幕1/4区域内显示一些个性化信息。再编写一个汇编语言的程序,作为用户程序,利用int 33、int 34、int 35和int 36产生中断调用你这4个服务程序。
	\item 扩充系统调用,实现三项以上新的功能,并编写一个测试所有系统调用功能的用户程序。
\end{itemize}

\section{实验环境}
% 包括:硬件或虚拟机配置方法、软件工具与作用、方案的思想、相关原理、程序流程、算法和数据结构、程序关键模块,结合代码与程序中的位置位置进行解释。不得抄袭,否则按作弊处理。
% 实验方案包括相关基础原理、实验工具和环境、程序流程和算法思想、数据结构与程序模块功能说明,代码文档组成说明等
具体环境选择原因已在实验一报告中说明。
\begin{itemize}
	\item Windows 10系统 + Ubuntu 18.04(LTS)子系统
	\item gcc 7.3.0 + nasm 2.13.02 + GNU ld (Binutils) 2.3.0
    \item GNU Make 4.1
	\item Oracle VM VirtualBox 5.2.8
    \item Bochs 2.6.9
	\item Sublime Text 3
\end{itemize}

虚拟机配置:内存4M,无硬盘,1.44M虚拟软盘引导。

\section{实验方案}
% 包括:主要工具安装使用过程及截图结果、程序过程中的操作步骤、测试数据、输入及输出说明、遇到的问题及解决情况、关键功能或操作的截图结果。不得抄袭,否则按作弊处理。


\section{实验总结}
% 每人必需写一段,文字不少于500字,可以写心得体会、问题讨论与思考、新的设想、感言总结或提出建议等等。不得抄袭,否则按作弊处理。


\section{参考资料}
\begin{enumerate}
	\item 李忠,王晓波,余洁,《x86汇编语言-从实模式到保护模式》,电子工业出版社,2013
	\item Interrupts, \url{https://wiki.osdev.org/Interrupts}
	\item 8259 PIC, \url{https://wiki.osdev.org/PIC#Programming_the_PIC_chips}
	\item 函数调用规则,\url{https://www.cs.princeton.edu/courses/archive/spr11/cos217/lectures/15AssemblyFunctions.pdf}
\end{enumerate}

\appendix
\appendixconfig
\section{程序清单}
\label{sec:code}
由于程序太多,请直接见压缩文件。

\section{附件文件说明}
\begin{center}
\begin{tabular}{|c|l|l|}\hline
序号 & 文件 & 描述 \\\hline
1 & \verb'bootloader.asm' & 主引导程序\\\hline
2 & \verb'os.asm' & 内核汇编部分\\\hline
3 & \verb'kernel.c' & 内核C部分\\\hline
4 & \verb'Makefile' & 编译指令文件\\\hline
5 & \verb'link.ld' & 链接文件\\\hline
6 & \verb'bochsrc.bxrc' & bochs调试文件\\\hline
7 & \verb'mydisk.img' & 核心虚拟软盘\\\hline
8$\thicksim$12 & \verb'prgX.asm' & 用户程序\\\hline
\end{tabular}
\end{center}

\end{document}

% 实验提交内容
% 实验报告:电子版(Word2003的DOC格式或PDF格式)
% 原程序文件及可执行代码程序文件
% 测试输入数据文件和输出数据文件
% 虚拟机软盘映像文件

% 基础实验项目5个和扩展实验7个
% 实验项目,迟交影响成绩评价!
% 工具与环境可由选择,开发新型工具或优化一套开发环境都可加分!
% 一系列基础实验项目必须连续完成,当前项目只能在前一个项目的基础上进行,体现出前后的进化关系,否则要被约谈,证明没有抄袭行为!
% 一个项目可提交多个改进的版本,实现新功能和个性化特征都有利于提高相应项目的成绩。
% 实验项目提交内容用winrar工具整体压缩打包,统一格式命名为:
%    <学号>+<姓名>+<实验项目号>+<版本号>.rar
%    姓名(学号)实验NvX.zip
%    实验报告、项目文件夹、映像文件
%    ftp://172.18.216.232 sysuac 下周六23:59

% 免考
% 条件:实验1~6全部评价AAAAB+B+或相当
% 最终成绩可能范围:75分以上