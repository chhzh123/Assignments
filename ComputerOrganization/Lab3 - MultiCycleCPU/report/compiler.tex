% !TEX root = main.tex

\subsection{简易编译器}
\qquad 本次实验用\verb'Python'写了一个简单的编译器来实现MIPS代码到二进制代码的转换。主要采用正则表达式库(\verb're')对输入的字符串进行处理。基本实现思路如下:
\begin{enumerate}
    \item 逐行读入汇编源文件,遇到非指令行(如注释)则直接丢弃,其余行送入\verb'parse'函数。
    \item 在\verb'parse'函数内判断指令名称,并依据不同的指令格式分别进行字符串的分割和处理,这里采用了正则表达式进行分割。
    \item 将操作、寄存器编号、立即数等全部转化为二进制。不够位数的用0补足,负数要用补码表示。
    \item 将这些二进制数连接起来,得到32位的二进制字符串,返回
    \item 逐行输出二进制编码和十六进制编码,并写入文件,注意这里采用小端存储(每8位作为一个字节/单位)
\end{enumerate}
\par 完整程序附在第\ref{sec:appendix}章编译器一节。