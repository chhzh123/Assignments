\documentclass[12pt,UTF8]{ctexart}
\usepackage{titling,setspace}
\usepackage{enumerate}
\usepackage{amsmath,amssymb,amsfonts}
\usepackage{listings}
\usepackage{comment}
\usepackage{float}
\usepackage{graphicx}
\usepackage{multicol,multirow}
\usepackage{fancyhdr} % 页眉
\usepackage{ctex,titlesec}
\usepackage[export]{adjustbox} % 图片左对齐
\usepackage[unicode=true,%本行非常重要 支持中文目录hyperref CJKbookmarks对二级目录没用
	colorlinks,
	linkcolor=black,
	anchorcolor=black,
	citecolor=black,
	CJKbookmarks=false]{hyperref}
\usepackage{xcolor}
\usepackage{geometry}
\geometry{top=25.4mm,bottom=25.4mm,left=31.8mm,right=31.8mm}
\pagestyle{plain}%删除页眉
\CTEXsetup[format={\large\bfseries}]{section}
\renewcommand\maketitlehooka{\null\mbox{}\vfill} % 标题页
\renewcommand\maketitlehookd{\vfill\null}

\newcommand{\chuhao}{\fontsize{42.2pt}{\baselineskip}\selectfont}
\newcommand{\xiaochu}{\fontsize{36.1pt}{\baselineskip}\selectfont}
\newcommand{\yihao}{\fontsize{26.1pt}{\baselineskip}\selectfont}
\newcommand{\xiaoyi}{\fontsize{24.1pt}{\baselineskip}\selectfont}
\newcommand{\erhao}{\fontsize{22.1pt}{\baselineskip}\selectfont}
\newcommand{\xiaoer}{\fontsize{18.1pt}{\baselineskip}\selectfont}
\newcommand{\sanhao}{\fontsize{16.1pt}{\baselineskip}\selectfont}
\newcommand{\xiaosan}{\fontsize{15.1pt}{\baselineskip}\selectfont}
\newcommand{\sihao}{\fontsize{14.1pt}{\baselineskip}\selectfont}
\newcommand{\xiaosi}{\fontsize{12.1pt}{\baselineskip}\selectfont}
\newcommand{\wuhao}{\fontsize{10.5pt}{\baselineskip}\selectfont}
\newcommand{\xiaowu}{\fontsize{9.0pt}{\baselineskip}\selectfont}
\newcommand{\liuhao}{\fontsize{7.5pt}{\baselineskip}\selectfont}
\newcommand{\xiaoliu}{\fontsize{6.5pt}{\baselineskip}\selectfont}
\newcommand{\qihao}{\fontsize{5.5pt}{\baselineskip}\selectfont}
\newcommand{\bahao}{\fontsize{5.0pt}{\baselineskip}\selectfont}
\titleformat{\section}{\xiaosi\bfseries}{\chinese{section}、 }{0em}{}
\titlespacing*{\section}{0pt}{3pt}{0pt}

\lstset{language=c++,basicstyle=\small,escapechar=`}
\setlength{\droptitle}{-100pt}%减少标题与页眉距离

\title{
\includegraphics[width=0.2\linewidth,left]{SYSU.png}~\\[1cm]
\textbf{\yihao 《计算机组成原理实验》\\\chuhao实验报告\\\quad\\\xiaoyi(实验一)}
}

\vspace{100pt}

\author{}
\date{}

\newcommand{\myuline}[1]{\begin{tabular}{c}#1\\\hline\end{tabular}}

\begin{document}

\clearpage\maketitle
\thispagestyle{empty}

\vspace{100pt}
\begin{spacing}{1.5}
\xiaoer
\begin{center}
\begin{tabular}{rc}
\makebox[6.2em][s]{\textbf{学\hspace{\fill}院\hspace{\fill}名\hspace{\fill}称}} : & 数据科学与计算机学院\\
\cline{2-2}\makebox[6.2em][s]{\textbf{专业(班级)}} : & 17级计科教务一班\\
\cline{2-2}\makebox[6.2em][s]{\textbf{学\hspace{\fill}生\hspace{\fill}姓\hspace{\fill}名}} :& 陈鸿峥\\
\cline{2-2}\makebox[6.2em][s]{\textbf{学\hspace{\fill}号}} : & 17341015\\
\cline{2-2}\makebox[6.2em][s]{\textbf{时\hspace{\fill}间}} : & 2018 年 10 月 8 日\\
\cline{2-2}
\end{tabular}
\end{center}
\end{spacing}

\newpage
\pagestyle{fancy}
\lhead{}
\rhead{\xiaowu\color{gray}{计算机组成原理实验}}
\setcounter{page}{1}
\quad\bigskip\bigskip
\rightline{\erhao\textbf{\underline{成绩:\qquad\qquad}}}
\vspace{20pt}
\leftline{\erhao\textbf{\underline{实验一:MIPS汇编语言程序设计实验}}}
\vspace{-10pt}

\wuhao
\section{实验目的}

\section{实验内容}
% 实验的具体内容与要求

\section{实验器材}
电脑一台,PCSpim仿真器软件一套

\section{实验过程与结果}
% 说明:根据需要书写相关内容,如:
% 程序流程图、设计的思想与方法、分析、实验步骤和实验结果及分析等。

\section{实验心得}
% 体会和建议。(必须认真写,若过于简单,扣分!)
% (所写内容,如实验的整个过程中,所碰到的问题、所思考的问题等,以及最后如何获得解决,从中得到什么?等等,当然,还可能存在未解决的问题,或有所建议等。整个来说,就是总结一下本实验的情况,温故而知新,这个道理显而易见。)

\section{程序清单}
\begin{lstlisting}
int main()
{
	return 0;
}
\end{lstlisting}


\end{document}