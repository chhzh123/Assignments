% !TEX root = main.tex

\newpage
\begin{multicols}{2}
\section{程序清单}
\label{sec:appendix}
\subsection{CPU主模块}
\begin{lstlisting}
timescale 1ns / 1ps

module CPU (
	input clk, reset,
	output RegDst,
	output ExtSel,
	output RegWrite, MemWrite,
	output ALUSrcA, ALUSrcB,
	output [2:0] ALUOp,
	output MemToReg,
	output Branch, Jump, Zero,
	output PCWrite,
	output [31:0] currPC, nextPC, instruction, alu_res,
	output wire [31:0] d1, d2, rsData, rtData, dbData,
	output wire [4:0] rs, rt, rd, sa
	);

	wire [5:0] opcode;
	wire [15:0] imm;
	wire [31:0] pc4;

	assign opcode = instruction[31:26];
	assign rs = instruction[25:21];
	assign rt = instruction[20:16];
	assign rd = instruction[15:11];
	assign sa = instruction[10:6];
	assign imm = instruction[15:0];

	PC pc(
		.clk(clk),
		.reset(reset),
		.PCWrite(PCWrite),
		.currPC(currPC),
		.nextPC(nextPC)
		);

	Adder add_pc4(
		.A(currPC),
		.B({{29{1'b0}},3'b100}),
		.res(pc4)
		);
	
	// instruction fetch (IF)
	InstructionMemory im(
		.address(currPC),
		.dataOut(instruction)
		);

	// instruction decode (ID)
	ControlUnit control(
		// input
		.opcode(opcode),
		.Zero(Zero),
		// output
		.RegDst(RegDst),
		.ExtSel(ExtSel),
		.RegWrite(RegWrite),
		.ALUSrcA(ALUSrcA),
		.ALUSrcB(ALUSrcB),
		.ALUOp(ALUOp),
		.MemToReg(MemToReg),
		.MemWrite(MemWrite),
		.Branch(Branch),
		.Jump(Jump),
		.PCWrite(PCWrite)
		);

	// execution (EXE)
	Registers reg_file(
		.clk(clk),
		.reset(reset),
		.r1(rs),
		.r2(rt),
		.wr(mux_regdst.res),
		.RegWrite(RegWrite),
		.wd(mux_memToReg.res)
		// d1 -> mux_srcA.A
		// d2 -> mux_srcB.A / dm.dataIn
		);

	assign d1 = mux_srcA.res;
	assign d2 = mux_srcB.res;
	assign rsData = reg_file.d1;
	assign rtData = reg_file.d2;

	MUX5 mux_regdst(
		.Sel(RegDst),
		.A(rt),
		.B(rd)
		// res -> reg_file.wr
		);

	MUX mux_srcA(
		.Sel(ALUSrcA),
		.A(reg_file.d1),
		.B({{27{1'b0}},sa})
		// res -> alu.A
		);
	
	Extender extender(
		.Sel(ExtSel),
		.dataIn(imm)
		// dataOut -> mux_srcB.B / sl_16
		);

	MUX mux_srcB(
		.Sel(ALUSrcB),
		.A(reg_file.d2),
		.B(extender.dataOut)
		// res -> alu.B
		);

	ALU alu(
		.op(ALUOp),
		.A(mux_srcA.res),
		.B(mux_srcB.res),
		// res -> dm.address / mux_memToReg.A
		.res(alu_res),
		.zero(Zero)
		);

	// access memory (MEM)
	DataMemory dm(
		.clk(clk),
		.reset(reset),
		.address(alu_res),
		.MemWrite(MemWrite),
		.dataIn(reg_file.d2)
		// dataOut -> mux_memToReg
		);

	// write back (WB)
	MUX mux_memToReg(
		.Sel(MemToReg),
		.A(alu_res),
		.B(dm.dataOut)
		// res -> reg_file.wd
		);

	assign dbData = mux_memToReg.res;

	// jump & branch
	ShiftLeft sl(
		.dataIn(extender.dataOut)
		// dataOut -> add_target.B
		);

	Adder add_target(
		.A(pc4),
		.B(sl.dataOut)
		// res -> mux_branch.B
		);

	MUX mux_branch(
		.Sel(Branch),
		.A(pc4),
		.B(add_target.res)
		// res -> mux_jump.A
		);

	MUX mux_jump(
		.Sel(Jump),
		.A(mux_branch.res),
		.B({pc4[31:28],instruction[25:0],2'b00}),
		.res(nextPC)
		);

endmodule
\end{lstlisting}

\subsection{指令存储器}
\begin{lstlisting}
module InstructionMemory (
	input [31:0] address,
	output [31:0] dataOut
	);
	
	reg [7:0] memory [0:255]; // 8 bits (bandwidth) * #256 (address)

	// initialization
	initial $readmemb("D:/instruction.data",memory);

	// output data (little endian)
	assign dataOut = {memory[address + 3],
					  memory[address + 2],
					  memory[address + 1],
					  memory[address]};

endmodule
\end{lstlisting}

\subsection{程序计数器(PC)}
\begin{lstlisting}
module PC (
    input clk,
    input reset,
    input PCWrite,
    input [31:0] nextPC,
    output reg [31:0] currPC
    );

    initial currPC = 0;

    always @(posedge clk or negedge reset) begin
        if (reset == 0)
            currPC <= 32'hFFFFFFFC;
        else if (PCWrite == 1)
            currPC <= nextPC;
    end

endmodule
\end{lstlisting}

\subsection{寄存器堆}
\begin{lstlisting}
module Registers (
    input clk,
    input reset,
    input [4:0] r1, // read reg #1 address
    input [4:0] r2, // read reg #2 address
    input [4:0] wr, // write reg address
    input RegWrite,
    input [31:0] wd, // write data
    output [31:0] d1, // read data 1
    output [31:0] d2 // read data 2
    );
    
    reg [31:0] register [0:31]; // 32 bits (bandwidth) * #32 (address)
    
    // initialization
    integer i;
    initial begin
        for (i = 0; i < 32; i = i + 1)
            register[i] = 0;
    end

    // read data
    assign d1 = (r1 == 0) ? 0 : register[r1];
    assign d2 = (r2 == 0) ? 0 : register[r2];

    // write data
    always @(negedge clk) begin
        if (reset == 0)
            register[0] <= 0;
        else if (wr != 0 && RegWrite == 1)
            register[wr] <= wd;
    end

endmodule
\end{lstlisting}

\subsection{数据存储器}
\begin{lstlisting}
module DataMemory (
    input clk,
    input reset,
    input [31:0] address,
    input MemWrite,
    input [31:0] dataIn, // write data
    output [31:0] dataOut // read data
    );
    
    reg [7:0] memory [0:255]; // 8 bits (bandwidth) * #256 (address)

    // initialization
    integer i;
    initial begin
        for (i = 0; i < 256; i = i + 1)
            memory[i] = 0;
    end

    // read data
    assign dataOut = (address == 0) ? 0 : {memory[address + 3],
                                           memory[address + 2],
                                           memory[address + 1],
                                           memory[address]};

    // write data
    always @(negedge clk) begin
        if (reset == 0)
            memory[0] <= 0;
        else if (MemWrite == 1 && address != 0) begin // do not use <=255!!!
            // little endian
            memory[address + 3] <= dataIn[31:24];
            memory[address + 2] <= dataIn[23:16];
            memory[address + 1] <= dataIn[15:8];
            memory[address] <= dataIn[7:0];
        end
    end

endmodule
\end{lstlisting}

\subsection{控制单元}
\begin{lstlisting}
module ControlUnit (
    input [5:0] opcode,
    input Zero,
    output reg RegDst,
    output reg ExtSel,
    output reg RegWrite,
    output reg ALUSrcA,
    output reg ALUSrcB,
    output reg [2:0] ALUOp,
    output reg MemToReg,
    output reg MemWrite,
    output reg Branch,
    output reg Jump,
    output reg PCWrite
    );

    always @ (opcode or Zero) begin // Zero!
    	RegDst   <= 0;
		ExtSel   <= 0;
		RegWrite <= 1;
		ALUSrcA  <= 0;
		ALUSrcB  <= 0;
		ALUOp    <= 3'b000;
		MemToReg <= 0;
		MemWrite <= 0;
		Branch   <= 0;
		Jump     <= 0;
		PCWrite  <= 1;
    	case (opcode)
    		6'b000000: begin // add rd, rs, rt
    				RegDst <= 1;
    				end
    		6'b000001: begin // sub rd, rs, rt
    				RegDst <= 1;
    				ALUOp <= 3'b001;
    				end
    		6'b000010: begin // addiu rt, rs, imm
                    ExtSel <= 1; // ???
    				ALUSrcB <= 1;
    				end
    		6'b010000: begin // andi rt, rs, imm
    				ALUSrcB <= 1;
    				ALUOp <= 3'b100;
    				end
    		6'b010001: begin // and rd, rs, rt
    				RegDst <= 1;
    				ALUOp <= 3'b100;
    				end
    		6'b010010: begin // ori rt, rs, imm
    				ALUSrcB <= 1;
    				ALUOp <= 3'b011;
    				end
    		6'b010011: begin // or rd, rs, rt
    				RegDst <= 1;
    				ALUOp <= 3'b011;
    				end
    		6'b011000: begin // sll rd, rt, sa
    				RegDst <= 1;
    				ALUSrcA <= 1;
    				ALUOp <= 3'b010;
    				end
    		6'b011100: begin // slti rt, rs, imm
                    ExtSel <= 1;
                    ALUSrcB <= 1; // remember!
    				ALUOp <= 3'b110;
    				end
    		6'b100110: begin // sw rt, imm(rs)
    				ExtSel <= 1;
    				RegWrite <= 0;
    				ALUSrcB <= 1;
    				MemWrite <= 1;
    				end
    		6'b100111: begin // lw rt, imm(rs)
    				ExtSel <= 1;
    				ALUSrcB <= 1;
    				MemToReg <= 1;
    				end
    		6'b110000: begin // beq rs, rt, imm	
					ExtSel <= 1;
    				RegWrite <= 0;
    				ALUOp <= 3'b001;
    				Branch <= Zero;
					end
			6'b110001: begin // bne rs, rt, imm	
					ExtSel <= 1;
    				RegWrite <= 0;
    				ALUOp <= 3'b001;
    				Branch <= ~Zero; // (rs - rt == 0) ? 1 : 0 Not equal!
					end
			6'b110010: begin // bltz rs, imm
					ExtSel <= 1;
    				RegWrite <= 0;
    				ALUOp <= 3'b110; // compare sign
    				Branch <= ~Zero; // a < 0 ? 1 : 0
					end
			6'b111000: begin // j addr
    				RegWrite <= 0;
    				Jump <= 1;
					end
			6'b111111: begin // halt
					PCWrite <= 0;
					end
		endcase
	end
    
endmodule
\end{lstlisting}

\subsection{算术逻辑单元(ALU)}
\begin{lstlisting}
module ALU (
    input [2:0] op,
    input [31:0] A,
    input [31:0] B,
    output reg [31:0] res,
    output zero
    );
    
    initial begin
        res = 0;
    end
    
    always @(op or A or B) begin
        case (op)
            3'b000: res = A + B;
            3'b001: res = A - B;
            3'b010: res = B << A; // B first!
            3'b011: res = A | B;
            3'b100: res = A & B;
            3'b101: res = (A < B) ? 1 : 0;
            3'b110: res = ((A < B && A[31] == B[31]) // both pos/neg num
                         || (A[31] == 1 && B[31] == 0)) // A neg B pos
                         ? 1 : 0; // not 8'h0000001 !!!
            3'b111: res = A ^ B;
        endcase
    end

    assign zero = (res == 0) ? 1 : 0;

endmodule
\end{lstlisting}

\subsection{多路选择器(MUX)}
\begin{lstlisting}
module MUX (
    input Sel,
    input [31:0] A,
    input [31:0] B,
    output reg [31:0] res
    );
    
    always @(Sel or A or B) begin
        res <= (Sel == 0) ? A : B;
    end

endmodule

module MUX5 (
    input Sel,
    input [4:0] A,
    input [4:0] B,
    output reg [4:0] res
    );
    
    always @(Sel or A or B) begin
        res <= (Sel == 0) ? A : B;
    end

endmodule
\end{lstlisting}

\subsection{数据扩展器}
\begin{lstlisting}
module Extender (
	input Sel,
	input [15:0] dataIn,
	output reg [31:0] dataOut
	);

	initial dataOut = 0;

	always @(Sel or dataIn) begin // dataIn!!!
		if (Sel == 0) // ZeroExt
			dataOut = {{16{1'b0}},dataIn[15:0]};
		else // SignExt
			dataOut = {{16{dataIn[15]}},dataIn[15:0]};
	end

endmodule
\end{lstlisting}

\subsection{仿真代码}
\begin{lstlisting}
module CPU_sim (
	output RegDst,
	output ExtSel,
	output RegWrite, MemWrite,
	output ALUSrcA, ALUSrcB,
	output [2:0] ALUOp,
	output MemToReg,
	output Branch, Jump, Zero,
	output PCWrite,
	output [31:0] currPC, nextPC, instruction, alu_res,
	output [4:0] rs, rt,
	output [31:0] d1, d2, rsData, rtData, dbData
	);
	
	reg clk;
	reg reset;

	CPU cpu(
		.clk(clk),
		.reset(reset),
		.RegDst(RegDst),
		.ExtSel(ExtSel),
		.RegWrite(RegWrite),
		.MemWrite(MemWrite),
		.ALUSrcA(ALUSrcA),
		.ALUSrcB(ALUSrcB),
		.ALUOp(ALUOp),
		.MemToReg(MemToReg),
		.Branch(Branch),
		.Jump(Jump),
		.Zero(Zero),
		.PCWrite(PCWrite),
		.currPC(currPC),
		.nextPC(nextPC),
		.instruction(instruction),
		.alu_res(alu_res),
		.rs(rs),
		.rt(rt),
		.d1(d1),
		.d2(d2),
		.rsData(rsData),
		.rtData(rtData),
		.dbData(dbData)
		);

	initial begin
		clk = 1;
		reset = 0;
		// wait for initialization
		#30;
		reset = 1;
	end

	always #50 clk = ~clk;

endmodule
\end{lstlisting}

\subsection{分频计数器}
\begin{lstlisting}
module Counter(
    input clr, // clear, say reset
    input clk, // original clock
    output reg [1:0] count_4,
    output reg clk_seg
    );
    
    // display 10kHz
    reg [16:0] count_dis; // 26 bits to store count: 2^17 > 10^5
    always @ (posedge clk or posedge clr)
    begin
        if (clr == 1) // reset
        begin
            clk_seg <= 0;
            count_dis <= 0;
        end
        else if (count_dis == 50_000 - 1) // return 0
        begin
            clk_seg <= ~clk_seg;
            count_dis <= 0;
        end
        else
        begin
            clk_seg <= clk_seg;
            count_dis <= count_dis + 1;
        end
    end

    always @ (posedge clk_seg or posedge clr)
    begin
        if (clr == 1 || count_4 == 4)
            count_4 <= 0;
        else
            count_4 <= count_4 + 1;
    end

endmodule
\end{lstlisting}

\subsection{七段数码管}
\begin{lstlisting}
module SegDisplay (
	input [3:0] data,
	output reg [6:0] dispcode
	);

	always @(data)
		case(data)
			// 0: on    1: off
			0: dispcode = 7'b000_0001; // remember!
			1: dispcode = 7'b100_1111;
			2: dispcode = 7'b001_0010;
			3: dispcode = 7'b000_0110;
			4: dispcode = 7'b100_1100;
			5: dispcode = 7'b010_0100;
			6: dispcode = 7'b010_0000;
			7: dispcode = 7'b000_1111;
			8: dispcode = 7'b000_0000;
			9: dispcode = 7'b000_0100;
			10: dispcode = 7'b000_1000; // A
			11: dispcode = 7'b110_0000; // b
			12: dispcode = 7'b011_0001; // C
			13: dispcode = 7'b100_0010; // d
			14: dispcode = 7'b001_0000; // e
			15: dispcode = 7'b011_1000; // F
		endcase

endmodule
\end{lstlisting}

\subsection{写板电路}
\begin{lstlisting}
module Show(
	input clk,
	input clk_cpu, // button
	input reset,
	input [1:0] SW_in,
	output reg [6:0] dispcode,
	output reg [3:0] out
	);

	// synchronize and reduce jitter
	reg in_detected = 1'b0;
	reg [15:0] inhistory = 16'h0000;
	always @(posedge clk) begin
		inhistory = {inhistory[15:0], clk_cpu};
		if (inhistory == 16'b0011111111111111)
			in_detected <= 1'b1;
		else
			in_detected <= 1'b0;
	end

	wire [1:0] seg_num; // not reg!

	Counter counter(
		.clk(clk),
		// output clock/counter
		.count_4(seg_num)
		);

	reg [31:0] firstNum;
	reg [31:0] secondNum;

	initial firstNum = 0;
	initial secondNum = 0;

	wire [31:0] currPC, nextPC, rsData, rtData, dbData, alu_res;
	wire [4:0] rs, rt;

	CPU cpu(
		// input
		.clk(in_detected),
		.reset(reset),
		// output
		.currPC(currPC),
		.nextPC(nextPC),
		.rs(rs),
		.rt(rt),
		.rsData(rsData),
		.rtData(rtData),
		.dbData(dbData),
		.alu_res(alu_res)
		);

	always @(SW_in) begin
		case (SW_in)
			2'b00: begin
				firstNum <= currPC;
				secondNum <= nextPC;
				end
			2'b01: begin
				firstNum <= {{27{1'b0}},rs};
				secondNum <= rsData;
				end
			2'b10: begin
				firstNum <= {{27{1'b0}},rt};
				secondNum <= rtData;
				end
			2'b11: begin
				firstNum <= alu_res;
				secondNum <= dbData;
				end
		endcase
	end

	SegDisplay seg1(
		.data(firstNum[7:4])
		// .dispcode
		);

	SegDisplay seg2(
		.data(firstNum[3:0])
		// .dispcode
		);

	SegDisplay seg3(
		.data(secondNum[7:4])
		// .dispcode
		);

	SegDisplay seg4(
		.data(secondNum[3:0])
		// .dispcode
		);

	always @ (seg_num or firstNum or secondNum)
		case (seg_num)
			0: begin
				out = 4'b1110;
				dispcode = seg4.dispcode;
				end
			1: begin
				out = 4'b1101;
				dispcode = seg3.dispcode;
				end
			2: begin
				out = 4'b1011;
				dispcode = seg2.dispcode;
				end
			3: begin
				out = 4'b0111;
				dispcode = seg1.dispcode;
				end
		endcase

endmodule
\end{lstlisting}

\subsection{限制文件(constraints.xdc)}
\begin{lstlisting}
set_property PACKAGE_PIN W5 [get_ports clk]
set_property PACKAGE_PIN T17 [get_ports clk_cpu]
set_property PACKAGE_PIN V17 [get_ports reset]
set_property PACKAGE_PIN R2 [get_ports {SW_in[1]}]
set_property PACKAGE_PIN T1 [get_ports {SW_in[0]}]
set_property PACKAGE_PIN W4 [get_ports {out[3]}]
set_property PACKAGE_PIN V4 [get_ports {out[2]}]
set_property PACKAGE_PIN U4 [get_ports {out[1]}]
set_property PACKAGE_PIN U2 [get_ports {out[0]}]
set_property PACKAGE_PIN W7 [get_ports {dispcode[6]}]
set_property PACKAGE_PIN W6 [get_ports {dispcode[5]}]
set_property PACKAGE_PIN U8 [get_ports {dispcode[4]}]
set_property PACKAGE_PIN V8 [get_ports {dispcode[3]}]
set_property PACKAGE_PIN U5 [get_ports {dispcode[2]}]
set_property PACKAGE_PIN V5 [get_ports {dispcode[1]}]
set_property PACKAGE_PIN U7 [get_ports {dispcode[0]}]

set_property IOSTANDARD LVCMOS33 [get_ports clk]
set_property IOSTANDARD LVCMOS33 [get_ports clk_cpu]
set_property IOSTANDARD LVCMOS33 [get_ports reset]
set_property IOSTANDARD LVCMOS33 [get_ports {SW_in[1]}]
set_property IOSTANDARD LVCMOS33 [get_ports {SW_in[0]}]
set_property IOSTANDARD LVCMOS33 [get_ports {out[3]}]
set_property IOSTANDARD LVCMOS33 [get_ports {out[2]}]
set_property IOSTANDARD LVCMOS33 [get_ports {out[1]}]
set_property IOSTANDARD LVCMOS33 [get_ports {out[0]}]
set_property IOSTANDARD LVCMOS33 [get_ports {dispcode[6]}]
set_property IOSTANDARD LVCMOS33 [get_ports {dispcode[5]}]
set_property IOSTANDARD LVCMOS33 [get_ports {dispcode[4]}]
set_property IOSTANDARD LVCMOS33 [get_ports {dispcode[3]}]
set_property IOSTANDARD LVCMOS33 [get_ports {dispcode[2]}]
set_property IOSTANDARD LVCMOS33 [get_ports {dispcode[1]}]
set_property IOSTANDARD LVCMOS33 [get_ports {dispcode[0]}]
\end{lstlisting}
\end{multicols}